\documentstyle[11pt]{article}
\setlength{\oddsidemargin}{12pt}
\setlength{\textwidth}{6.5in}
\setlength{\textheight}{9in}
\pagestyle{empty}
\setlength{\parskip}{7pt plus 2pt minus 2pt}

\begin{document}

\begin{center}
{{\large CS 330 : Discrete Computational Structures}}\\

%\vspace*{1cm}

{\bf Fall Semester, 2015}\\

{\sc Assignment \#7}\\
{\bf Due Date:}  Sunday, Oct 25\\
Cahlen Brancheau
\end{center}

\noindent {\bf Suggested Reading:} Rosen Section 5.3; LLM Chapter 6.1 - 6.3

These are the problems that you need to turn in. For more
practice, you are encouraged to work on the other problems. {\bf Always
explain your answers and show your reasoning.}

\begin{enumerate}

\item {\bf [12 Pts]} Let $S$ defined recursively by (1) $5 \in S$ and (2) if $s \in S$ and $t \in S$, then $st \in S$. 
Let $A = \{5^i \mid i \in {\cal Z}^+\}$. Prove that

\begin{enumerate}

\item {\bf [6 Pts]} $A \subseteq S$ by mathematical induction.\\
\textbf{Answer}:\\

Basis: $5^1 =  5$ and by definition $5 \in S$

Inductive Step:\\
Assume $5^k \in S$ where $k \in \cal Z$\\
Prove: $5^{k+1} \in S$\\
$5^{k+1} = 5^k * 5$, $5^k \in S$ by the IH and $5 \in S$ by step (1) in the recursive definition of $S$. Therefore $5^k * 5 \in S$ by step (2) in the recursive definition of $S$.\\\\
Quod erat demonstrandum.\\

\item {\bf [6 Pts]} $S \subseteq A$ by structural induction.\\
\textbf{Answer}:\\

Basis: $5 \in S$ by step (1) in the recursive definition of $S$ and $5 = 5^1 \in A$

Inductive Step:\\
Let $a,b \in S$, $a * b \in S$ by step (2) in the recursive definition of $S$.\\
Assume $a,b \in A$\\
Prove: $a * b \in A$\\
By the IH $a = 5^x$ and $b = 5^y$ where $x,y \in \cal Z$. $a * b = 5^x * 5^y = 5^{x+y}$. $x + y \in \cal Z$ therefore $a*b \in A$.\\\\
Quod erat demonstrandum.\\

\end{enumerate}

\clearpage

\item {\bf [12 Pts]} Let $S$ be defined by (1) $(0,0) \in S$, and (2) if $(a,b) \in S$, 
then $(a,b+5) \in S$, $(a+1,b+4) \in S$ and $(a+2,b+3) \in S$.

\begin{enumerate}

\item {\bf [6 Pts]} Use structural induction to prove that if $(a,b) \in S$ then $5$ divides $a + b$.\\
\textbf{Answer}:\\

Basis: $(0,0) \in S$ and $5$ divides $0+0=0$\\

Inductive Step:\\
Assume $(a,b) \in S$ where $5$ divides $a + b$
$(x,y) \in S$ and let $(a,b) \in S$ where $(a,b)$ is the relation that the recursive definition of $S$ was applied to in order to get to $(x,y)$.\\\\
It must be the case that $x = a, y = b+5$ or $x = a+1, y = b+4$ or $x = a+2, y = b+3$.\\

\textbf{Case 1}:  $x = a, y = b+5$. $x + y = a + b + 5$. By the IH $5$ divides $a + b$ and $5$ divides $5$.\\
\textbf{Case 2}:  $x = a+1, y = b+4$. $x + y = a + b + 5$. By the IH $5$ divides $a + b$ and $5$ divides $5$.\\
\textbf{Case 2}:  $x = a+2, y = b+3$. $x + y = a + b + 5$. By the IH $5$ divides $a + b$ and $5$ divides $5$.\\\\
Quod erat demonstrandum.\\

\item {\bf [6 Pts]} Disprove the converse of the statement above, {\it i.e.}, show that if $a,b \in \cal N$, and $a+b$ is divisible by 5, it does not follow that $(a,b) \in S$. Modify the recursive definition of $S$ to make the converse true.\\
\textbf{Answer}:\\

$5$ divides $3+2 = 5$, but $(3,2) \notin S$\\

New Definition of $S$\\

(1) $(0,0) \in S$\\
(2) if $(a,b) \in S$, then 
$(a,b+5) \in S$, 
$(a+1,b+4) \in S$, 
$(a+2,b+3) \in S$ 
$(a+3,b+2) \in S$,  
$(a+4,b+1) \in S$,  
$(a+5,b) \in S$.\\\\
Quod erat demonstrandum.\\

\end{enumerate}

\clearpage

\item {\bf [6 Pts]} Give a recursive definition of the set of bit strings that are palindromes.\\
\textbf{Answer}:\\
Basis: An empty string is a palindrome and string with a one bit length is a palindrome.\\

Inductive Step:\\
A strings in the form $pcp$ are palindromes, where $c$ is a bit and $p$ is a palindrome.\\\\
Quod erat demonstrandum.\\


\item {\bf [20 Pts]} 

\begin{enumerate}

\item {\bf [8 Pts]} Give an inductive definition of the set 
    $L = \{(a,b) \mid a,b \in {\cal Z}, (a-b) \, {\rm mod} \, 3 = 0\}$.
Let $L'$ be the set obtained by your inductive definition.\\
\textbf{Answer}:\\

Basis: $(0,0) \in L'$\\

Recursive: if $(x,y) \in L'$ then $(x+1,y+1) \in L'$, $(x-1,y-1) \in L'$, $(x,y+3) \in L'$, $(x,y-3) \in L'$.\\

\item {\bf [6 Pts]}	Prove that $L' \subseteq L$.\\
\textbf{Answer}:\\

By definition of $L' \subseteq L$ all pairs in $L'$ must be divisible by $3$. In other words for any $(a,b) \in L', a-b = 3k$ where $k \in \cal Z$\\
With the recursive definition of $L'$ the possible options are thus:\\\\

\textbf{Case 1}: $(a-1, b-1)$, this is the same as $a-b$ therefore still divisible by 3.\\
\textbf{Case 2}: $(a+1, b+1)$, this is the same as $a-b$ therefore still divisible by 3.\\
\textbf{Case 3}: $(a, b+3)$, $a - b + 3 = 3k - 3 = 3(k-1)$ therefore still divisible by 3.\\
\textbf{Case 4}: $(a, b-3)$, $a - b - 3 = 3k + 3 = 3(k+1)$ therefore still divisible by 3.\\


\item {\bf [6 Pts]}	Prove that $L \subseteq L'$.\\
\textbf{Answer}:\\
		
All pairs in $L$ must be divisible by $3$. In other words for any $(a,b) \in L, a-b = 3k \rightarrow b = 3k - a$ where $k \in \cal Z$.\\
Any element in $L$ with be in the form $(a, 3k-a)$. To get to any other pair using the definition of $L'$ we apply the following.\\

(1) $(0,0)$ to $(0, 3k)$. If $k \geq 0$ apply $(x, y+3)$ $k$ times, else apply $(x, y-3)$ $k$ times.\\
(2) $(0, 3k)$ to $(a, 3k - a)$. If $a \geq 0$ apply $(x+1, b+1)$ $a$ times, else apply $(x-1, y-1)$ $a$ times.
			
\end{enumerate}

\end{enumerate}
\end{document}
