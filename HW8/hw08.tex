\documentclass[11pt]{article}
\usepackage{amssymb}
\setlength{\oddsidemargin}{12pt}
\setlength{\textwidth}{6.5in}
\setlength{\textheight}{9in}
\pagestyle{empty}
\setlength{\parskip}{7pt plus 2pt minus 2pt}

\begin{document}

\begin{center}
{{\large CS 330 : Discrete Computational Structures}}\\

%\vspace*{1cm}

{\bf Fall Semester, 2015}\\

{\sc Assignment \#8}\\
{\bf Due Date:}  Tuesday, Nov 3\\
Cahlen Brancheau
\end{center}

\noindent {\bf Suggested Reading:} Rosen 9.1 and 9.5, LLM 9.4

These are the problems that you need to turn in. For more
practice, you are encouraged to work on the other problems. {\bf Always
explain your answers and show your reasoning.}

%\vspace*{0.5cm}

\begin{enumerate}

\item {\bf [18 Pts]} For each of these relations decide whether it is reflexive, anti-reflexive, symmetric, anti-symmetric and transitive. Justify your answers. $R_1$ and $R_2$ are over the set of real numbers and $R_3$ is over the set of all people.

\begin{enumerate}

\item $(x,y) \in R_1$ if and only if $xy = 4$\\
\textbf{Reflexive}: No. $(1,1) \notin R_1$\\
\textbf{Anti-Reflexive}: No. $(2,2) \in R_1$\\
\textbf{Symmetric}: Yes. $xy = 4 \leftrightarrow xy = 4$ therefore $(x,y) \in R_1 \rightarrow (y,x) \in R_1$\\
\textbf{Anti-Symmetric}: No. Because Symmetry\\
\textbf{Transitive}: No. $(1,4) \in R_1$ and $(4,1) \in R_1$ but $(1,1) \notin R_1$\\


\item $(x,y) \in R_2$ if and only if $x = 1$ or $y = 1$\\
\textbf{Reflexive}: No. $(5,5) \notin R_2$\\
\textbf{Anti-Reflexive}: No. $(1,1) \in R_2$\\
\textbf{Symmetric}: Yes. $(x,y) \in R_2 \rightarrow (y,x) \in R_2$\\
\textbf{Anti-Symmetric}: No. $(2,1) \in R_2$ and $(1,2) \in R_2$\\
\textbf{Transitive}: No. $(2,1) \in R_2$ and $(1,2) \in R_2$ but $(2,2) \notin R_2$\\


\item $(x,y) \in R_3$ if and only if $x$ and $y$ share a common biological parent\\
\textbf{Reflexive}: Yes. Trivially true. $(x,x) \in R_3$\\
\textbf{Anti-Reflexive}: No. $(x,x) \in R_3$\\
\textbf{Symmetric}: Yes. $(x,y) \in R_3 \rightarrow (y,x) \in R_3$\\
\textbf{Anti-Symmetric}: No. Because symmetry\\
\textbf{Transitive}: No. Bob and Linda have baby $a$. Linda and Steve have baby $b$. Steve and Beth have baby $c$. $(a,b) \in R_3$ and $(b,c) \in R_3$ but $(a,c) \notin R_3$


\end{enumerate}

\clearpage

\item {\bf [12 Pts]} Consider relation $R_4$ on the set of positive real numbers where 
$(x,y) \in R_4$ if and only if $x/y \in \cal Q$. Decide whether it is (a) reflexive, anti-reflexive, symmetric, anti-symmetric and transitive and (b) show that this an equivalence relation. Describe the equivalence classes. (c) What is the equivalence class of $2$? of $\pi$? Justify your answers.\\\\
(a) 
\textbf{Reflexive}: Yes. $\frac{a}{a} = 1$ and $1 \in \mathbb{Q}$ therefore $(a,a) \in R_4$\\
\textbf{Anti-Reflexive}: No. Because reflexivity\\
\textbf{Symmetric}: Yes. $\forall x,y \in \mathbb{R^+}, \frac{x}{y} \in \mathbb{Q} \rightarrow \frac{y}{x} \in \mathbb{Q}$ therefore $(x,y) \in R_4$ and $(y,x) \in R_4$\\
\textbf{Anti-Symmetric}: No. Because Symmetry\\
\textbf{Transitive}: Yes. $(x,y) \in R_4 \rightarrow (y,x) \in R_4$ and $(x,x) \in R_4$\\

(b) 
An equivalence relation is defined as a relation that is \textbf{Reflexive}, \textbf{Symmetric}, and \textbf{Transitive}. $R_4$ is all three. $R_4$ is an equivalence relation.\\

(c) 
$[2]_R = \{ \forall n \in \mathbb{R+} | \frac{2}{n} or \frac{n}{2}\}$, $[\pi]_R = \{ \forall n \in \mathbb{R+} | \frac{\pi}{n} or \frac{n}{\pi}\}$\\

\item {\bf [12 Pts]} Let $R_5$ be the relation on $\mathbb{Z} \times \mathbb{Z}$ where 
$((a,b), (c,d)) \in R_5$ if and only if $a-c = b-d$. \\

\begin{enumerate}

\item Prove that $R_5$ is an equivalence relation. \\

\textbf{Reflexive}: Yes. $a - a = b - b \rightarrow ((a,b)(a,b)) \in R_5$\\
\textbf{Symmetric}: Yes. $a - c = b - d \leftrightarrow c - a = d - b$ therefore $((a,b)(c,d)) \in R_5$ and $((c,d)(a,b)) \in R_5$\\
\textbf{Transitive}: Yes. $((a,b)(c,d)) \in R_5$ and $((c,d)(e,f)) \in R_5$ gives us $a - c = b - d$ and $c - e = d - f$. $(a-c) - (c - f) = (b - d) - (d - f)$ which implies that $a-e = b - f$ therefore $((a,b)(e,f)) \in R_5$\\

\item Define a function $f$ such that $f(a,b) = f(c,d)$ if and only if $((a,b), (c,d)) \in R$.\\

$f(x,y) = x - y$\\

\item Define the equivalence class containing $(1,1)$ and list 2 elements in the class.\\

$[(1,1)] = \{ (x,x) | x \in  \mathbb{Z}\}$\\

Elements: $(2,2)$, $(3,3)$\\

\item Describe each equivalence class. How many classes are there and how many elements in each class?\\

There are a countably infinite number of classes each with a countably infinite number of elements.

\end{enumerate}

\clearpage

\item {\bf [4 Pts]} Describe all the equivalence classes for the relation {\it congruence modulo 6} over $\cal Z$, using set-builder notation.\\
$[0] = \{ 6n | n \in \mathbb{Z} \}$\\
$[1] = \{ 6n + 1 | n \in \mathbb{Z} \}$\\
$[2] = \{ 6n + 2 | n \in \mathbb{Z} \}$\\
$[3] = \{ 6n + 3 | n \in \mathbb{Z} \}$\\
$[4] = \{ 6n + 4 | n \in \mathbb{Z} \}$\\
$[5] = \{ 6n + 5 | n \in \mathbb{Z} \}$\\

\item {\bf [4 Pts]} LLM Problem 9.9 (c)\\
$8.01 \rightarrow 8.02 \rightarrow 18.01 \rightarrow 18.02 \rightarrow 6.042 \rightarrow 6.046 \rightarrow 6.840 \rightarrow 18.03 \rightarrow 6.002 \rightarrow 6.001 \rightarrow 6.034 \rightarrow 6.003 \rightarrow 6.004 \rightarrow 6.033 \rightarrow 6.857$\\

\end{enumerate}
\end{document}

