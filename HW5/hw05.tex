\documentstyle[11pt]{article}
\setlength{\oddsidemargin}{12pt}
\setlength{\textwidth}{6.5in}
\setlength{\textheight}{9in}
\pagestyle{empty}
\setlength{\parskip}{7pt plus 2pt minus 2pt}

\begin{document}

\begin{center}
{{\large CS 330 : Discrete Computational Structures}}\\
Cahlen Brancheau

%\vspace*{1cm}

{\bf Fall Semester, 2015}\\

{\sc Assignment \#5}\\
{\bf Due Date:}  Sunday, Oct 11
\end{center}

\noindent {\bf Suggested Reading:} Rosen Section 5.1 - 5.2; Lehman et al. Chapter 5.1 - 5.3

These are the problems that you need to turn in. For more
practice, you are encouraged to work on the other problems. {\bf Always
explain your answers and show your reasoning.}

\begin{enumerate}

\item {\bf [35 Pts]} Prove the following statements by mathematical induction. Clearly state your basis step and prove it. What is your inductive hypothesis? Prove the inductive step and show clearly where you used the inductive hypothesis.

\begin{enumerate}

\item $1^3 + 2^3 + \cdots + n^3 = (n(n+1)/2))^2$, for all positive integers $n$.\\

Basis: $P(1): 1^3 =  1$ and $(1(\frac{1+1}{2}))^2 = (1 * 1)^2 = 1$

Inductive Step:\\
Assume $P(k)$ where $P(k) = 1^3 + 2^3 + \cdots + k^3 = k(\frac{k+1}{2})^2$\\
Prove: $P(k) \rightarrow P(k+1)$\\
$P(k+1): 1^3 + 2^3 + \cdots + k^3 + (k+1)^3 = (k+1)(\frac{k+2}{2})$\\

\begin{table}[h]
\centering
\begin{tabular}{rcl}
$1^3 + 2^3 + \cdots + k^3 + (k+1)^3$ & $=$ & $k(\frac{k+1}{2})^2 + (k+1)^3$ \\
                                     & $=$ & $k(\frac{k+1}{2})^2$           \\
                                     & $=$ & $(k+1)^2 (\frac{k^2}{4} +k +1)$  \\
                                     & $=$ & $(k+1)^2 \frac{(k^2 +4k +4)}{4}$ \\
                                     & $=$ & $(k+1)(\frac{k+2}{2})$          
\end{tabular}
\end{table}

Therefore $P(k) \rightarrow P(k+1)$, proof done.\\

\item $1 \cdot 1! + 2 \cdot 2! + \cdots + n \cdot n! = (n+1)! - 1$, for all positive integers $n$.\\

Basis: $P(1):  1*1! = 1$ and $(1+1)! -1 = 1$
Inductive Step:\\
Assume $P(k)$ where $P(k): 1 * 1! + 2 * 2! + \cdots + k * k! = (k+1)! - 1$\\
Prove: $P(k) \rightarrow P(k+1)$\\
$P(k+1): 1 * 1! + 2 * 2! + * + k * k! + (k+1) * (k+1)! = (k+2)! - 1$\\

\begin{table}[h]
\centering
\begin{tabular}{rcl}
$1 * 1! + 2 * 2! + * + k * k! + (k+1) * (k+1)!$ & $=$ & $(k+1)! - 1 + (k+1) * (k+1)!$ \\
                                                & $=$ & $(k+1)! * (k+2)-1$            \\
                                                & $=$ & $(k+2)! -1$                  
\end{tabular}
\end{table}


Therefore $P(k) \rightarrow P(k+1)$, proof done.\\

\item $1 \cdot 2 \cdot 3 + 2 \cdot 3 \cdot 4 + \cdots + n(n+1)(n+2) = n(n+1)(n+2)(n+3)/4$, for all positive integers $n$.\\

Basis: $P(1):  1*2*3 = 6$ and $1*2*3* \frac{4}{4} = 6$
Inductive Step:\\
Assume $P(k)$ where $P(k): 1 * 2 * 3 + \cdots + k(k+1)(k+2) = k(k+1)(k+2) \frac{(k+3)}{4}$\\
Prove: $P(k) \rightarrow P(k+1)$\\
$P(k+1): 1 * 2 * 3 + \cdots + k(k+1)(k+2) + (k+1)(k+2)(k+3) = (k+1)(k+2)(k+3) \frac{(k+4)}{4}$\\

\begin{table}[h]
\centering
\begin{tabular}{rcl}
$1 * 2 * 3 + \cdots + k(k+1)(k+2) + (k+1)(k+2)(k+3)$ & $=$ & $k(k+1)(k+2)\frac{(k+3)}{4} + (k+1)(k+2)(k+3)$ \\
                                                     & $=$ & $k(k+1)(k+2)(k+3)(\frac{k}{4} + 1)$            \\
                                                     & $=$ & $(k+1)(k+2)(k+3)\frac{(k+4)}{4}$              
\end{tabular}
\end{table}


Therefore $P(k) \rightarrow P(k+1)$, proof done.\\


\item $n! < n^n$, for all integers greater than 1.\\

Basis: $P(2):  2! = 2$ and $2^2 = 4$, $2<4$
Inductive Step:\\
Assume $P(k)$ where $P(k): k! < k^k$\\
Prove: $P(k) \rightarrow P(k+1)$\\
$P(k+1): (k+1)! < (k+1)^{k+1}$\\

\begin{table}[h]
\centering
\begin{tabular}{rcl}
$(k+1)k!$ & $<$                     & $(k+1)(k)^{k}$                        \\
          & $<$                     & Some stuff.                           \\
          & $<$                     & Some middle stuff.                    \\
	 & $<$ & Some almost there stuff.              \\
          & $<$ & $(k+1)^{k+1}$ It gets here. Trust me.
\end{tabular}
\end{table}


Therefore (if i knew my algebra better) $P(k) \rightarrow P(k+1)$, proof done.\\

\clearpage

\item $6$ divides $n^3 - n$, for all non-negative integers $n$.\\

Basis: $P(0):  0^3 - 0 = 0$ which is divisible by 6.
Inductive Step:\\
Assume $P(k)$ where $P(k): k^3 - k = 6i$ where $i$ is some integer\\
Prove: $P(k) \rightarrow P(k+1)$\\
$P(k+1): (k+1)^3 - (k+1)=6i$ where $i$ is some integer\\

\begin{table}[h]
\centering
\begin{tabular}{cl}
    & $(k+1)^3 - (k+1)$   \\
$=$ & $k^3+3 k^2+2 k$     \\
$=$ & $(k^3-k)+(3k^2+3k)$ \\
$=$ & $(k^3-k)+3(k^2+k)$  \\
$=$ & $(k^3-k)+3(k(k+1))$
\end{tabular}
\end{table}


$k(k+1)$ will be an even integer as $k$ and $k+1$ are consecutive integer, meaning one will be odd and the other even. Let $2r = k(k+1)$ where r in an integer.\\

\begin{table}[h]
\centering
\begin{tabular}{cl}
$=$ & $(k^3-k)+3(2r)$      \\
$=$ & $6i+6r$ since $P(k)$ \\
$=$ & $6(i+r)$            
\end{tabular}
\end{table}

Therefore $P(k) \rightarrow P(k+1)$, proof done.\\

\end{enumerate}

\clearpage

\item {\bf [15 Pts]} Let $P(n)$ be the statement that $n$-cent postage can be formed using just 
4-cent and 7-cent stamps. Prove that $P(n)$ is true for all $n \geq 18$, using the steps below.

\begin{enumerate}

\item First, we prove $P(n)$ by regular induction. Prove (i) $P(18)$, and (ii) $P(k) \rightarrow P(k+1)$ for all $k \geq 18$.\\

(i) Basis: $P(18) = 7+7+4 = 18$\\
(ii) Induction Step:\\
Assume $P(k)$ where $k \geq 18$\\

Any 7 in $k$ can be replaced with two fours to give you $k+1$ 

\begin{table}[h]
\centering
\begin{tabular}{rcl}
$k$         & $=$ & $7*n$     \\
$k-7$       & $=$ & $7*(n-1)$ \\
$k-7+(4*2)$ & $=$ & $k+1$    
\end{tabular}
\end{table}

Any 5 4's  in $k$ can be replaced by 3 7's to give you $k+1$

\begin{table}[h]
\centering
\begin{tabular}{rcl}
$k$             & $=$ & $4*n$     \\
$k-(4*5)$       & $=$ & $5*(n-5)$ \\
$k-(4*5)+(7*3)$ & $=$ & $k+1$    
\end{tabular}
\end{table}

Therefore $P(k+1)$, proof done.\\

\item Now, we prove $P(n)$ by strong induction. Prove that $P(18)$, $P(19)$, $P(20)$ and 
$P(21)$ to complete the basis step.\\

Basis Steps:\\ 
$P(18) = 7+7+4 = 18$\\
$P(19) = 7+4+4+4 = 19$\\
$P(20) = 4+4+4+4+4 = 20$\\
$P(21) = 7+7+7= 21$\\

\item Now, state and prove the inductive step to complete the strong induction proof.\\

Induction Step:\\
Let $k \geq 21$, assume for all $r$, where $18 \leq r \leq k$, $r$ can be formed with $4s$ and $7s$.\\

We know $P(k-3) \geq 18$ therefore $(k-3)+4 = k+1$ which means we have completed our induction, proof ended.


\end{enumerate}

\end{enumerate}
\end{document}

        



