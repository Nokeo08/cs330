\documentstyle[12pt]{article}
\setlength{\oddsidemargin}{12pt}
\setlength{\textwidth}{6.5in}
\setlength{\textheight}{9in}
\pagestyle{empty}
\setlength{\parskip}{7pt plus 2pt minus 2pt}

\begin{document}

\begin{center}
{{\large CS 330 : Discrete Computational Structures}}\\
Cahlen Brancheau Homework 1

\vspace*{1cm}

{\bf Fall Semester, 2015}\\

{\sc Assignment \#1}\\
{\bf Due Date:}  Friday, September 4
\end{center}

\noindent {\bf Suggested Reading:} Rosen Sections 1.1 - 1.3; Lehman et al. Sections 1.1,  3.1 - 3.4

These are the problems that you need to hand in for grading. For more
practice, you are encouraged to work on the other problems. Always
explain your answers and show your reasoning.

\begin{enumerate}

\item {\bf [10 Pts]} Let $p$, $q$ and $r$ be the propositions ``It is snowing'', ``It is freezing'' and ``Classes are cancelled'' respectively. Express each of these propositions as an English sentence.

p = It is snowing, q = It is freezing, r = Classes are cancelled

\begin{enumerate}
\item $p \wedge \neg q$
\\answer: It is snowing and it is not freezing.
\item $\neg p \rightarrow \neg q$
\\answer: If it is not snowing then it is not freezing.
\item $p \vee q \rightarrow r$ 
\\answer: If it is snowing or it is freezing then classes are cancelled.
\item $p \leftrightarrow q$
\\answer: It is snowing if and only if it is freezing.
\item $(p \wedge \neg r) \rightarrow \neg q$
\\answer: If it is snowing and classes are not cancelled then it is not freezing.
\end{enumerate}

\item {\bf [10 Pts]} Write each of these statements in the form ``if p then q''.

\begin{enumerate}
\item It is necessary to attend all lectures to get an A in the class.
\\answer: If you want to get and A then you must attend all lectures.
\item Ann will get caught whenever she cheats.
\\answer: If Ann cheats then she will get caught.
\item You will finish the race unless it gets too hot.
\\answer: Unless it gets to hot then you will finish the race
\item Eating candy everyday is a sufficient condition for getting tooth decay.
\\answer: If I eat candy every day, then my teeth will decay.
\item I will miss lecture only if I am sick.
\\answer: If I am sick then I will miss class.\\\\
\end{enumerate}

\clearpage

\item {\bf [5 Pts]} Show that $(p \vee q) \wedge (\neg p \vee r) \rightarrow (q \vee r)$ is a tautology.
\begin{table}[h]
\centering
\begin{tabular}{|c|c|c|c|c|c|c|c|c|}
\hline
$ \neg p$ & $p$ & $q$ & $r$ & $p \vee q$ & $ \neg p \vee r$ & $(p \vee q) \wedge ( \neg p \vee r)$ & $q \vee r$ & $(p \vee q) \wedge ( \neg p \vee r) \rightarrow (q \vee r)$ \\ \hline
F         & T   & T   & T   & T          & T                & T                                    & T          & T                                                           \\ \hline
F         & T   & T   & F   & T          & F                & F                                    & T          & T                                                           \\ \hline
F         & T   & F   & T   & T          & T                & T                                    & T          & T                                                           \\ \hline
F         & T   & F   & F   & T          & F                & F                                    & F          & T                                                           \\ \hline
T         & F   & T   & T   & T          & T                & T                                    & T          & T                                                           \\ \hline
T         & F   & T   & F   & T          & T                & T                                    & T          & T                                                           \\ \hline
T         & F   & F   & T   & F          & T                & F                                    & T          & T                                                           \\ \hline
T         & F   & F   & F   & F          & T                & F                                    & F          & T                                                           \\ \hline
\end{tabular}
\end{table}

\item {\bf [15 Pts]} On an island, there are three kinds of people, knights, knaves and spies. Knights always tell the truth, knaves always lie and spies can do either. You meet three people, A, B and C, where one is a knight, one is a knave and one is a spy. For the following problems, state all solutions (there may be no unique solution) or state that there are no solutions. Explain your reasoning.

\begin{enumerate}
\item A says `I am the knight', B says `I am the knave' and C says `I am the spy'.
\\\\Answer: A: Knight. B: Spy. C: Knave. A, B, and C all make claims about themselves, according to the rules of each role A must me the Knight, B cannot be the Knave so is therefore is the Spy, and C can only be the Knave.\\
\item A says `I am the knave', B says `A is not a knave' and C says `I am a knight'.
\\\\Answer: A: Spy. B: Knight. C: Knave. A's statement cannot be true, therefore A is the Spy. B's statement is true, therefore B cannot be the Knave and must be the Knight. C's statement is false, meaning that C is the Knave.\\
\item A says `C is the spy', B says `I am not the spy' and C says `I am not the spy'.
\\\\Answer: A: Knave. B: Spy. C: Knight. B and C both claim not to be the Spy, the Knave must lie so cannot be B or C therefore must be A. If A is the Knave then C is not the Spy and must be the Knight. B can only be the Spy.\\
\end{enumerate}

\clearpage

\item {\bf [10 Pts]} Prove that $(p \rightarrow r) \wedge (q \rightarrow r)$ and $(p \vee q) \rightarrow r$ are logically equivalent by (a) truth tables, and (b) by deduction using the logical equivalences studied in class (don't use this logical equivalence!).
    
\begin{table}[h]
\centering
\begin{tabular}{|c|c|c|c|c|c|c|l|c|}
\hline
$p$ & $q$ & $r$ & $p \rightarrow r$ & $q \rightarrow r$ & $(p \rightarrow r) \wedge (q \rightarrow r)$ & $p \vee q$ & $r$ & $(p \vee q) \rightarrow r$ \\ \hline
T   & T   & T   & T                 & T                 & T                                            & T          & T   & T                          \\ \hline
T   & T   & F   & F                 & F                 & F                                            & T          & F   & F                          \\ \hline
T   & F   & T   & T                 & T                 & T                                            & T          & T   & T                          \\ \hline
T   & F   & F   & F                 & T                 & F                                            & T          & F   & F                          \\ \hline
F   & T   & T   & T                 & T                 & T                                            & T          & T   & T                          \\ \hline
F   & T   & F   & T                 & F                 & F                                            & T          & F   & F                          \\ \hline
F   & F   & T   & T                 & T                 & T                                            & F          & T   & T                          \\ \hline
F   & F   & F   & T                 & T                 & T                                            & F          & F   & T                          \\ \hline
\end{tabular}
\end{table}

\begin{table}[h]
\centering
\begin{tabular}{lclc}
$(p \rightarrow r) \wedge (q \rightarrow r)$ & $\equiv$ & $(p \vee q) \rightarrow r$           &                                             \\
$(\neg p \vee r) \wedge (\neg q \vee r)$     & $\equiv$ & $\neg (p \vee q) \vee r$        	     & Logical Equivalence Table 6 \\
$(\neg p \wedge \neg q) \vee r$              	& $\equiv$ & $(\neg p \wedge \neg q) \vee r$ & Distributive and De Morgan                          
\end{tabular}
\end{table}

\end{enumerate}
\end{document}


