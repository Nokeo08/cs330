\documentclass[11pt]{article}
\setlength{\oddsidemargin}{12pt}
\setlength{\textwidth}{6.5in}
\setlength{\textheight}{9in}
\pagestyle{empty}
\setlength{\parskip}{7pt plus 2pt minus 2pt}
\usepackage{mathtools}
\DeclarePairedDelimiter\ceil{\lceil}{\rceil}
\DeclarePairedDelimiter\floor{\lfloor}{\rfloor}

\begin{document}

\begin{center}
{{\large CS 330 : Discrete Computational Structures}}\\

%\vspace*{1cm}

{\bf Fall Semester, 2015}\\

{\sc Assignment \#10}\\
{\bf Due Date:}  Sunday, Nov 22\\
Cahlen Brancheau
\end{center}

\noindent {\bf Suggested Reading:} Rosen Sections 6.1 - 6.3.

These are the problems that you need to turn in. Always explain your answers and show your reasoning. 
{\bf Spend time giving a complete solution. You will be graded based on how well you explain your answers. 
Just correct answers will not be enough!}

%\vspace*{0.5cm}

\begin{enumerate}

\item {\bf [4 Pts]} An ISU Computer Science shirt is sold in 6 colors, 5 sizes, striped or solid, and long sleeve or short sleeve. (a) How many different shirts are being sold? (b) What if the black and yellow shirts only come in short-sleeve and solid?\\
\textbf{Answer}:\\
\textbf{(a)} Product Rule: $6 \times 5 \times 2 \times 2 = 120$ different shirts.\\\
\textbf{(b)} Yellow and Black: $2 \times 5 \times 1 \times 1 = 20$\\
The rest: $4 \times 5 \times 2 \times 2 = 80$\\
Total: $20 + 80 = 100$ different shirts.\\

\item {\bf [6 Pts]} How many integers between 10000 and 99999, inclusive, are divisible by 3 or 5 or 7?\\
\textbf{Answer}:\\
Number divisible by $3$: $\floor*{\frac{99999}{3}} - \floor*{\frac{10000-1}{3}} = 30000$\\\\
Number divisible by $5$: $\floor*{\frac{99999}{5}} - \floor*{\frac{10000-1}{5}} = 18000$\\\\
Number divisible by $7$: $\floor*{\frac{99999}{7}} - \floor*{\frac{10000-1}{7}} = 12857$\\\\
Number divisible by $3 \times 5$: $\floor*{\frac{99999}{3 \times 5}} - \floor*{\frac{10000-1}{3 \times 5}} = 6000$\\\\
Number divisible by $3 \times 7$: $\floor*{\frac{99999}{3 \times 7}} - \floor*{\frac{10000-1}{3 \times 7}} = 4285$\\\\
Number divisible by $5 \times 7$: $\floor*{\frac{99999}{5 \times 7}} - \floor*{\frac{10000-1}{5 \times 7}} = 2572$\\\\

The total is: $30000 + 18000 + 12857 - 6000 - 4285 - 2572 =  48000$\\
 
 \clearpage
 
\item {\bf [10 Pts]} Let $A$ and $B$ be sets of 7 elements and 10 elements, respectively.\\\\
        \textbf{(a)} How many different functions possible from $A$ to $B$? from $B$ to $A$?\\
\textbf{Answer}:\\
Every element in A has $10$ choices, therefore there are $10^7$ different functions.\\

        \textbf{(b)} How many different relations possible from $A$ to $B$?\\
\textbf{Answer}:\\
The are $7 \times 10 = 70$ pairs so there are $2^{70}$ relations.\\

        \textbf{(c)} How many of the functions from $A$ to $B$ are one-to-one?\\
\textbf{Answer}:\\
There are 10 elements for the first choice, 9 elements for the second and so on so the answer is $P(10, 7) = \frac{10!}{(10-7)!} = 604800$\\

        \textbf{(d)} How many of the functions from $B$ to $A$ are onto?\\
\textbf{Answer}:\\\\






\item {\bf [6 Pts]} In how many ways can a photographer arrange 7 people in a row from a family of 10 people, if\\\\
\textbf{(a)} Mom and Dad are in the photo\\
\textbf{Answer}:\\ 
First choose 5 people from 8 (total excluding mom and dad) and them line them up with mom and dad. $8 \choose 5$ $\times 8! = 2257920$ different ways.\\\\
\textbf{(b)} Mom and Dad are next to each other in the photo\\
\textbf{Answer}:\\ 
There are $2!$ ways to arrange mom and dad to if they're treated like one person. Then we still pick 5 from the 8. Then we arrange the rest. $2! \times $ $8 \choose 5$ $\times 6! = 80640$ different ways\\\\
\textbf{(c)} either Mom or Dad is in the photo, not both.\\
\textbf{Answer}:\\
First pick mom or Dad $2!$. Then pick the rest $8 \choose 6$. Then arrange them $7!$. $2! \times$ $8 \choose 6$ $\times 7! = 282240$\\\\

\clearpage





\item {\bf [6 Pts]} A sack contains 40 movie tickets, 5 for each of 8 different movies. Five friends want to go to a movie. How many tickets would you have to remove from the sack to guarantee that everyone will be able to watch the same movie? What principle did you use?\\
\textbf{Answer}:\\
In the worst case $4 \times 8 +1 = 33$ tickets. It could be the case that I draw 4 tickets for every one of the 8 movies so by the Pigeonhole Principal if I draw 1 more I am guaranteed to have 5 tickets for some movie.\\

 What if everyone wants to go to `Spectre'? How many tickets would you have to remove from the sack in that case?\\
\textbf{Answer}:\\

Trick question! We would not want to see Spectre...\\

In the worst case $7 \times 5 + 5 = 40$ tickets. It could be the case that I draw all of the tickets for all of the other movie before "Spectre', so the worst case is that I draw all of them.\\\\



\item {\bf [6 Pts]} How many bit strings of length 7 contain\\
\textbf{(a)} exactly three 1s?\\
\textbf{Answer}:\\ 
Choose exactly 3 positions from the 7 to be a 1. $ 7 \choose 3$ $= 35$ strings.\\

\textbf{(b)}  at most three 1's?\\
\textbf{Answer}:\\ 
Add the number of exactly 3, exactly 2, exactly 1, and exactly 0 together. $7 \choose 3$$+$$7 \choose 2$$+$$7 \choose 1$$+$$7 \choose 0$$= 64$ strings.\\

\textbf{(c)} at least three 1's?\\
\textbf{Answer}:\\
Similar to the above, add the number of exactly 3, exactly 4, exactly 5, exactly 6, and exactly 7 together. $7 \choose 3$$+$$7 \choose 4$$+$$7 \choose 5$$+$$7 \choose 6$$+$$7 \choose 7$$= 99$ strings.\\


\item {\bf [6 Pts]} A coin is flipped nine times where each flip comes up either head or tails. How many possible outcomes contain at least five heads? Can you come up with two different ways to do this problem? How about three?\\
1) Do it like it like the previous question. $9 \choose 5$$+$$9 \choose 6$$+$$9 \choose 7$$+$$9 \choose 8$$+$$9 \choose 9$$ = 256$ ways.\\
2) You could also count the tails instead of the heads $9 \choose 4$$+$$9 \choose 3$$+$$9 \choose 2$$+$$9 \choose 1$$+$$9 \choose 0$$ = 256$ ways.\\

\clearpage

\item {\bf [6 Pts]} 12 women and 10 men are on the faculty. How many ways are there to pick a committee of 6 if\\ 
\textbf{(a)} Claire and Jane will not serve together\\
\textbf{Answer}:\\  
$22 \choose 6$$-$$22 \choose 4$$= 67298$ The total number of committees minus the when Claire and Jane are together.\\

\textbf{(b)} at least one woman must be chosen\\
\textbf{Answer}:\\ 
$22 \choose 6$$-$$10 \choose 6$$= 74403$ The total number of committees minus the number of ways without a women.\\ 


\textbf{(c)} at least one man and one woman must be chosen\\
\textbf{Answer}:\\
$22 \choose 6$$-$$($$12 \choose 6$$+$$10 \choose 6$$)$$= 73479$ The total number of committees minus the number of ways with only women plus the number of ways with only men.\\ 

Are there multiple ways to solve these problems? Explain.\\
\textbf{Answer}:\\
In stead of subtracting the the bad cases you could count the good cases. Just like I did in problem 7.

\end{enumerate}
\end{document}

