\documentclass[12pt]{article}
\setlength{\oddsidemargin}{12pt}
\setlength{\textwidth}{6.5in}
\setlength{\textheight}{9in}
\pagestyle{empty}
\setlength{\parskip}{7pt plus 2pt minus 2pt}

\usepackage{amssymb}

\begin{document}

\begin{center}
{{\large CS 330 : Discrete Computational Structures}}\\
Cahlen Brancheau HW 3

{\bf Fall Semester, 2015}\\

{\sc Assignment \#3}\\
{\bf Due Date:}  Sunday, September 20
\end{center}

\noindent {\bf Suggested Reading:} Rosen Sections 1.7 - 1.8, 2.1; Lehman et al. Chapter 1 and 4.1

These are the problems that you need to turn in. For more
practice, you are encouraged to work on the other problems. Always
explain your answers and show your reasoning.

\begin{enumerate}

\item {\bf [5 Pts]} Prove that the average of two rational numbers is rational. Use a direct proof.\linebreak \linebreak
ANSWER: Let $p, q$ be rational numbers and $p = \frac{a}{b}$ and $q = \frac{c}{d}$ and $b,d \neq 0$ and $a,b,c,d$ are all integers.\\
$\frac{p+q}{2} = (\frac{a}{b} + \frac{c}{d}) \frac{1}{2} = \frac{ad+bc}{2bd}$, $ad, bc, 2bd$ are all integers so the result is rational.

\item {\bf [6 Pts]} Prove that there is an irrational number between any rational number and irrational number.\linebreak \linebreak
ANSWER: Let $r$ be an irrational number, also let $p$ be a rational number where $p = \frac{a}{b}$ where $a,b$ are integers and $b \neq 0$.\\
Assume $\frac{a}{b} < r$ therefore $\frac{a}{b} < \frac{\frac{a}{b}}{2} + \frac{r}{2} < r$, $\frac{\frac{a}{b}}{2}$ is rational and $\frac{r}{2}$ is irrational.
The sum of a rational and and irrational is itself irrational.\\
Proof: Suppose that $\frac{a}{b} + r$ is rational then $\frac{a}{b} + r = \frac{m}{n}$
\begin{table}[h]
\centering
\begin{tabular}{ccc}
$\frac{a}{b} + r$ & $=$ & $\frac{m}{n}$        \\
$r$               & $=$ & $\frac{mb - na}{nb}$ 
\end{tabular}
\end{table}\linebreak
This is a contradiction because by definition an irrational cannot be represented as a ratio or two integers. Therefore the sum must result in an irrational number.


\item {\bf [6 Pts]} Prove that the square root of 15 is irrational.\linebreak \linebreak
ANSWER: Proof by contradiction. Assume that $\sqrt{15}$ is rational.
$\sqrt{15} = \frac{a}{b}, b \neq 0$ and $a,b$ have no common divisors and $a, b$ are integers.
\begin{table}[h]
\centering
\begin{tabular}{ccc}
$\sqrt{3}$           & $=$                  & $\frac{a}{b}$        \\
\multicolumn{1}{l}{} & \multicolumn{1}{l}{} & \multicolumn{1}{l}{} \\
$3$                  & $=$                  & $\frac{a^2}{b^2}$    \\
\multicolumn{1}{l}{} & \multicolumn{1}{l}{} & \multicolumn{1}{l}{} \\
$3b^2$               & $=$                  & $a^2$
\end{tabular}
\end{table}\\
So $a^2$ is divisible by 3 and $a$ is divisible by 3 $\leftrightarrow a = 3c$
\begin{table}[h]
\centering
\begin{tabular}{ccc}
$\sqrt{3}$           & $=$                  & $\frac{a}{b}$        \\
\multicolumn{1}{l}{} & \multicolumn{1}{l}{} & \multicolumn{1}{l}{} \\
$3$                  & $=$                  & $\frac{a^2}{b^2}$    \\
\multicolumn{1}{l}{} & \multicolumn{1}{l}{} & \multicolumn{1}{l}{} \\
$3b^2$               & $=$                  & $a^2$
\end{tabular}
\end{table}\\
So $b^2$ is divisible by 3 and $b$ is divisible by 3.\\
So $a and b$ have a common divisor which is a contradiction.


\item {\bf [6 Pts]} Prove that if you have 25 cookies that you will distribute to 6 children, at least one child will get more than 4 cookies. What proof method did you use?\linebreak \linebreak
ANSWER: If each child is given 4 cookies, the maximum without going over, then we have distributed $4*6=24$ cookies. This means that one child will necessarily have a 5th cookie. This is a direct proof.

\item {\bf [6 Pts]} Prove or disprove that $p$ is odd if and only if $p^3$ is odd.\linebreak \linebreak
ANSWER: This must be proved in two separate proofs.\\
Prove $p$ is odd $\rightarrow$ $p^3$ is odd\\
If $p = 2k+1$ (where k is some integer) then $p^3 = (2k+1)^3 = 8k^3 + 12k^ + 6k +1 = 2(4k^3 + 6k^2 + 6)+1$ Because $k$ is an integer $4k^3 + 6k^2 + 6$ must also be an integer. Therefore $p^3$ is also odd.\\

Prove $p^3$ is odd $\rightarrow$ $p$ is odd\\
The contrapositive statement is $p$ is even $\rightarrow$ $p^3$ is even.
If $p = 2k$ (where k is some integer) then $p^3 = (2k)^3 = 8k^3= 2(4k^3)$ Because $k$ is an integer $4k^3$ must also be an integer. Therefore $p^3$ is also even.\\

Therefore $p$ is odd $\leftrightarrow$ $p^3$ is odd is true.

\item {\bf [6 Pts]} Prove that there exist 1000 consecutive positive numbers that are not perfect squares. Is your proof constructive or non-constructive? Explain.\linebreak \linebreak
ANSWER: $1000^2$ is a perfect square, $1001^2$ is the next perfect square and the difference between them is $(1001^2)-(1000^2)=2001$. Which means that there is more than 1000 numbers without a perfect square. This proof is constructive because I constructed a concrete example that demonstrates the the numbers between $1001^2$ and $1000^2$ cannot be perfect squares.

\item {\bf [5 Pts]} Prove that if $A$ is a subset of $C$ and $B$ is a subset of $D$, then $A \times B$ is a subset of $C \times D$.\\\\
ANSWER: Let $x \in A, y \in B$ therefore $(x,y) \in A \times B$. Since $A \subseteq C$ and $B \subseteq D$ then $x \in C, y \in D$ Therefore $(x,y) \in C \times D$. Therefore $A \times B \subseteq C \times D$.

\clearpage
\item {\bf [4 Pts]} Use set builder notation to describe the following sets:\\
       (a) $\{6, 14, 22, \ldots\}$ (b) $\{3, 6, 11, 18, \ldots\}$\linebreak \linebreak
       ANSWER:\\
(a) $\{6+8p \mid p \in \mathbb{N}\}$ (b) $\{p^2+2 \mid p \in \mathbb{P}\}$

\item {\bf [6 Pts]} Prove that if $A$ and $B$ are non-empty sets where $A \neq B$, then $A \times B \neq B \times A$.\linebreak \linebreak
       ANSWER: Suppose $A \times B = B \times A$\\
       Let $x \in A$ and $y \in B$ We then have $(x,y) \in A \times B = B \times A$, By definition of the Cartesian product we have $x \in B, y \in A$, which is to say $A \subseteq B$ and $B \subseteq A$, meaning $A=B$. This contradicts the assumption that $A \neq B$, meaning that if $A$ and $B$ are non-empty sets where $A \neq B$, then $A \times B \neq B \times A$.

\end{enumerate}
\end{document}