\documentclass[11pt]{article}
\usepackage{amssymb}
\setlength{\oddsidemargin}{12pt}
\setlength{\textwidth}{6.5in}
\setlength{\textheight}{9in}
\pagestyle{empty}
\setlength{\parskip}{7pt plus 2pt minus 2pt}

\begin{document}

\begin{center}
{{\large CS 330 : Discrete Computational Structures}}\\

%\vspace*{1cm}

{\bf Fall Semester, 2015}\\

{\sc Assignment \#9}\\
{\bf Due Date:}  Tuesday, Nov 17\\
Cahlen Brancheau
\end{center}

\noindent {\bf Suggested Reading:} Rosen Section 2.5; LLM Chapter 7.1

These are the problems that you need to turn in. For more
practice, you are encouraged to work on the other problems. {\bf Always
explain your answers and show your reasoning.}

\begin{enumerate}

\item {\bf [16 Pts]} Show that the following sets are countably 
  infinite, by defining a bijection between $\cal N$ (or $\cal Z^+$) and that
  set. You do not need to prove that your function is bijective.

\begin{enumerate}

\item {\bf [8 Pts]} the set of positive integers divisible by 5\\
\textbf{Answer}:\\

$f$: $Z^+ \rightarrow \cal S$, $f(x) = 5x$, which is a bijection.\\

\textbf{Onto}: Assume $m \in \cal S$, so $m = 5k$ where $k \in \cal Z^+$. $f(k) = 5k = m$. $\square$

\textbf{One-To-One}: Assume $f(m) = f(n)$, where $m, n \in \cal Z^+$, $5m = 5n \rightarrow m = n$. $\square$\\

\item {\bf [8 Pts]} $\{1,2,3\} \times {\cal Z}$

\end{enumerate}

\item {\bf [16 Pts]} Determine whether the following sets are countable or
  uncountable. Prove your answer.\\
\textbf{Answer}:\\



\begin{enumerate}

\item {\bf [8 Pts]} the set of real numbers with decimal representation
  consisting of all 1's ($1.11$ and $11.111 \ldots$ are such numbers).\\
\textbf{Answer}:\\

Let $A = $ the set defined above.\\

First restrict the problem to the positive reals because there is a disjoint union of the positive and negative values and there is a bijection between them.\\

Let $S = $ the set of all positive values in $A$.\\\\

$f: \cal N \times \cal N$ $\rightarrow \cal S:$ $(n,m) \rightarrow \sum_{i=1}^{n} 10^i + \sum_{i=1}^{m} \frac{1}{10^i}$\\

$(n,m)$ maps to $n$ $1's$ to the left of the dot and $m$ $1's$ of the right side. $f$ is a bijection, $\cal N \times \cal N$ is countable and $\cal S$ is countable therefore $A$ is also countable.\\\\

\item {\bf [8 Pts]} the set of real numbers with decimal representation
  consisting of 1's and 9's\\
\textbf{Answer}:\\ \\
Let us enumerate the values between $0$ and $1$.\\

$0.\textbf{1}111111191199...$\\
$0.9\textbf{1}19191199191... $\\
$0.19\textbf{9}9119199119... $\\
$0.199\textbf{1}111911911... $\\
$0.9191\textbf{9}19199111... $\\
$0.91919\textbf{1}1911199... $\\
$...$\\

Taking the numbers bolded we get $0.119191$. This could not appear in our list because it would differ by at least one digit. We could not possibly enumerate all of the values between $0$ and $1$, so this set in uncountable. 

\end{enumerate}


\item {\bf [9 Pts]} Give an example of two uncountable sets $A$ and $B$ such that $A \cap B$ is (a) finite, (b) countably infinite, (c) uncountably infinite.\\ 
\textbf{Answer}:\\

1)

Let $A = \{ 1 \leq x \leq 2 | x \in \cal R \}$\\
Let $B = \{ 2 \leq x \leq 3 | x \in \cal R \}$\\

$A$ and $B$ are uncountable and $A \cap B = 2 \rightarrow$ finite. $\square$ \\

2)

Let $A = \cal Q$ $\cup$ $\cal R^-$\\
Let $B = \cal Q$ $\cup$ $\cal R^+$\\

$A$ and $B$ are uncountable and $A \cap B = \cal Q \rightarrow$ countably infinite. $\square$ \\

3)

Let $A = \{ 0 \leq x \leq 2 | x \in \cal R \}$\\
Let $B = \{ 1 \leq x \leq 3 | x \in \cal R \}$\\

$A$ and $B$ are uncountable and $A \cap B = [0,1] \rightarrow$ uncountably infinite. $\square$ \\
 
\clearpage
 
\item {\bf [9 Pts]} Prove that the set of functions from $\cal N$ to $\{0,1\}$ is
  uncountable, by using a diagonalization argument.\\
  \textbf{Answer}:\\
  
  Assume $\cal N$ $\rightarrow$ $\{0,1\}$ is countable and can be enumerated by $\cal S =$ $\{f_1, f_2,...\}$.\\
  
  $f(x) = 0$ if $f_x(x) = 1$, $1$ if $f_x(x) = 0$.\\
  
  So $f \neq f_x$ for any value $x$, because $f$ differs from $f_x$ in the $nth$ position. The enumeration of the functions cannot completely list all of the functions. $\square$

\end{enumerate}
\end{document}

\end{enumerate}
\end{document}
